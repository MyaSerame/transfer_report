\setlength{\parskip}{\baselineskip}
\begin{abstract}
    
Pneumonia is an illness usually caused by an acute respiratory tract infection in which lungs become inflamed and congested, limiting oxygen intake and leading to breathlessness. Despite the reduction in cases among children under the age of 5, pneumonia is still the main cause of childhood death in low income countries in Asia and Africa. The global community has therefore committed to ending preventable mortality and morbidity caused by pneumonia by 2025. To achieve this goal, the World Health Organisation (WHO) developed a pneumonia control strategy which provides clinical case management guidelines to offer a basic standard for appropriate assessment and treatment of sick children in Low and Middle Class-Income Countries (LMIC). However, due to factors such as inadequate training and supervision, as well as a shortage of skilled health professionals, adherence to these strategies remain poor in many settings. This failure in guideline implementation is further compounded by large numbers of seriously ill children in need of care.

The WHO recommends the use of raised respiratory rate (RR) and chest wall in-drawing to help health workers in developing countries to diagnose pneumonia. A particular problem, however, is the accuracy and reliability of identification of these key clinical features of pneumonia by junior clinicians. Guidelines on diagnosis, hospital admission and ultimately treatment require clinicians to measure respiratory rate by counting chest wall movements and assess the degree of respiratory distress by physical examination. Studies have shown poor inter-observer agreement for assessing clinical signs, thus technologies that can accurately identify and standardise assessment of respiratory distress would be a major advance towards improving the efficiency and quality of clinical assessment and, ultimately, clinical outcomes for pneumonia.

Smartphones offer a platform for developing these technologies to help healthcare professionals consistently identify clinical signs of severe illness, improve diagnosis of pneumonia and assess its severity. Smart phone cameras have the potential to enable standardised recognition of clinical signs of respiratory distress based on image processing and machine learning algorithms. These advances, combined with low-cost sensors (e.g. for oxygen saturation or temperature measurement), could result in a new generation of decision support tools. The same technology may also contribute towards refining existing WHO clinical algorithms amid growing concerns of diminished specificity of current guidelines for reliably identifying bacterial pneumonia in the post-pneumococcal and Haemophilus I vaccine era. 

This transfer report covers the development of algorithms for estimation of heart rate (HR) and RR across two clinical datasets. In chapter 3, red and near-infrared signals from a wrist-worn photoplethysmography (PPG) device were used to estimate HR (r = 0.99, MAE 1.2 = beats/minute) and RR (r = 0.67, MAE = 1.6 breaths/minute) during periods of induced hypoxia in a clinical study involving healthy volunteers. In chapter 4, video data from an observational neonatal intensive care unit (NICU) study was used to estimate HR (r = 0.83, MAE = 3.2 ) and RR (r = 0.71, MAE = 5.3). These results show that the algorithms developed are applicable in realistic scenarios using data from both wearables and video cameras.

\end{abstract}